\documentclass{scrartcl}

\usepackage[utf8]{inputenc}
\usepackage[T1]{fontenc}

% PYTHON	
\usepackage{xcolor}
\usepackage{listings}
\usepackage{setspace}
\lstset{ 
basicstyle=\tiny,
  backgroundcolor=\color{white},   % choose the background color; you must add \usepackage{color} or \usepackage{xcolor}; should come as last argument
  breakatwhitespace=false,         % sets if automatic breaks should only happen at whitespace
  breaklines=true,                 % sets automatic line breaking
  captionpos=b,                    % sets the caption-position to bottom
  commentstyle=\color{green},    % comment style
  deletekeywords={...},            % if you want to delete keywords from the given language
  escapeinside={\%*}{*)},          % if you want to add LaTeX within your code
  extendedchars=true,              % lets you use non-ASCII characters; for 8-bits encodings only, does not work with UTF-8
  keepspaces=true,                 % keeps spaces in text, useful for keeping indentation of code (possibly needs columns=flexible)
  keywordstyle=\color{blue},       % keyword style
  language=Octave,                 % the language of the code
  morekeywords={*,...},            % if you want to add more keywords to the set
  numbers=left,                    % where to put the line-numbers; possible values are (none, left, right)
  numbersep=5pt,                   % how far the line-numbers are from the code
  numberstyle=\tiny\color{gray}, % the style that is used for the line-numbers
  rulecolor=\color{black},         % if not set, the frame-color may be changed on line-breaks within not-black text (e.g. comments (green here))
  showspaces=false,                % show spaces everywhere adding particular underscores; it overrides 'showstringspaces'
  showstringspaces=false,          % underline spaces within strings only
  showtabs=false,                  % show tabs within strings adding particular underscores
  stringstyle=\color{cyan},     % string literal style
  tabsize=2,	                   % sets default tabsize to 2 spaces
  title=\lstname                   % show the filename of files included with \lstinputlisting; also try caption instead of title
}




\usepackage{hyperref}
\usepackage[utf8]{inputenc}
\usepackage[T1]{fontenc}
\usepackage{lmodern}
\usepackage{amsmath}
\usepackage{comment}

\title{Spherical Pendulum with Python}
\author{Arun Leander}
\date{2019}

\begin{document}

\maketitle


\section{Introduction}

Simulate a spherical pendulum with python.

\section {Equations}

The spherical pendulums position can be described with two coordinates. $\theta$ is the angle in a plane which includes the Z-axis and $\phi$ is the angle on the plane perpendicular to the Z-axis. With $l$ as the length of the pendulum, the position in cartesian coordinates is

\begin{align}
x& = l \sin{\theta} \cos{\phi}                 \\
y& = l \sin{\theta} \sin{\phi}                 \\
z& = -l \cos{\theta}                                
\end{align}


The energies.

\begin{align}
E_{kin} &= T = \frac{1}{2} mv^2                 \\
E_{pot} &= V = mgh
\end{align}

The Lagrangian.
\begin{equation}
L = T - V
\end{equation}

To calculate the kinetic energy $T$, I have to replace $v^2$.

\begin{equation}
v^2 =   \dot{x}^2 + \dot{y}^2 + \dot{z}^2    
\end{equation}


\subsection{Velocity}
\label{ssec:velocity}

$v^2$ can be written using $\theta$ and $\phi$ only. This requires the derivates in respect to the change of time $t$.

% x
\begin{align}
\begin{split}
\dot{x}^2& = (\frac{\partial x}{\partial t})^2			\\
& = (\frac{\partial }{\partial t} l \sin{\theta} \cos{\phi})^2	\\
& = l^2(\sin{\theta} (-\sin{\phi}) \dot{\phi} + \cos{\theta}\cos{\phi} \dot{\theta})^2				\\
& = l^2 [
	\sin^2\theta \sin^2 \phi \dot{\phi}^2 + 
	\cos^2\theta \cos^2\phi \dot{\theta}^2 -
	2 \sin\theta\cos\theta\dot{\theta}\sin{\phi}\cos\phi\dot{\phi}
	]
%y 
\end{split}
\\
\begin{split}
\dot{y}^2& = (\frac{\partial y}{\partial t})^2			\\
& = (\frac{\partial }{\partial t} l \sin{\theta} \sin{\phi})^2	\\
& = l^2(\sin{\theta} \cos\phi \dot{\phi} + \cos\theta\sin{\phi} \dot{\theta})^2				\\
& = l^2 [
	\sin^2\theta \cos^2 \phi \dot\phi^2 + 
	\cos^2\theta \sin^2\phi \dot{\theta}^2 + 
	2 \sin{\theta} \cos\theta \dot{\theta} \sin{\phi}\cos\phi \dot{\phi}
	]
\end{split}
\\
%z
\begin{split}
\dot{z}^2& = (\frac{\partial z}{\partial t})^2			\\
& = (\frac{\partial }{\partial t} - l\cos{\theta} )^2	\\
& = l^2 ( \sin^2\theta \dot{\theta}^2 )
\end{split}
\end{align}

\begin{align}
%x+y+z
\begin{split}
\frac{v^2}{l^2}& = (\sin^2\theta \sin^2 \phi \dot{\phi}^2 + 
	\cos^2\theta \cos^2\phi \dot{\theta}^2 -
	2 \sin\theta\cos\theta\dot{\theta}\sin{\phi}\cos\phi\dot{\phi})\\
&\phantom{{}={}} + (\sin^2\theta \cos^2 \phi \dot\phi^2 + 
	\cos^2\theta \sin^2\phi \dot{\theta}^2 + 
	2 \sin{\theta} \cos\theta \dot{\theta} \sin{\phi}\cos\phi \dot{\phi} ) \\
&\phantom{{}={}} +	\sin^2\theta \dot{\theta}^2
\end{split}
\\
\begin{split}
\frac{v^2}{l^2}& = \sin^2\theta \sin^2 \phi \dot{\phi}^2 + \sin^2\theta \cos^2 \phi \dot\phi^2 \\
&\phantom{{}={}} + \cos^2\theta \cos^2\phi \dot{\theta}^2 +\cos^2\theta \sin^2\phi \dot{\theta}^2  \\
&\phantom{{}={}} +	\sin^2\theta \dot{\theta}^2
\end{split}
\\
\begin{split}
\frac{v^2}{l^2}& = \sin^2\theta \dot{\phi}^2 (\sin^2 \phi +\cos^2 \phi) + \cos^2\theta \dot{\theta}^2 (\cos^2\phi + \sin^2\phi) \\
&\phantom{{}={}} +	\sin^2\theta \dot{\theta}^2
\end{split}
\\
\begin{split}
\frac{v^2}{l^2}& = \sin^2\theta \dot{\phi}^2 + (\cos^2\theta + \sin^2\theta) \dot{\theta}^2
\end{split}
\\
\begin{split}
v^2& = l^2 (\sin^2\theta \dot{\phi}^2 + \dot{\theta}^2)
\end{split}
\end{align}

Now the energies and the lagrangian can be written using $\theta$ and $\phi$.

\begin{align}
T& = \frac{1}{2} m l^2 (\sin^2\theta \dot{\phi}^2 + \dot{\theta}^2)               \\
V_g& = - m g l \cos\theta								\\
V_f& = m l (f_z \cos\theta - (f_x \cos\phi + f_y sin(\phi)) \sin(\theta))								              \\
L& = \frac{1}{2} m  l^2 (\sin^2\theta \dot{\phi}^2 + \dot{\theta}^2) + m g l \cos\theta	    \\
L_f& = \frac{1}{2} m  l^2 (\sin^2\theta \dot{\phi}^2 + \dot{\theta}^2) - m l (f_z \cos\theta - (f_x \cos\phi + f_y sin(\phi)) \sin(\theta))
\end{align}

I inserted a generalized force $f$. For gravity use $f = \left(\begin{smallmatrix}0\\0\\-10\end{smallmatrix}\right)$.

\subsection {Relative Velocity}
\label{ssec:rvelocity}
The velocity relative to a moving system can be expressed with $v_w = v - w$ where $w$ is the velocity of the system.

\begin{align}
  {v_w}^2& = (\dot{x} - w_x)^2 + (\dot{y} - w_y)^2 + (\dot{z} - w_z)^2 \\
  \begin{split}
  {v_w}^2 &= 
      (-l \sin{\theta} \sin{\phi} \dot{\phi} + l \cos{\theta}\cos{\phi} \dot{\theta} - w_x)^2 
      \\ &\qquad + (l \sin{\theta} \cos\phi \dot{\phi} + l \cos\theta\sin{\phi} \dot{\theta} - w_y)^2
    + (l \sin\theta \dot{\theta} - w_z)^2 
  \end{split}
\end{align}








\subsection{Second Lagrange Equation}

The second Lagrange equation for each coordinate $q_i$ is
\begin{equation}
\frac{\partial }{\partial t} \frac{\partial L}{\partial \dot{q}_i} - \frac{\partial L}{\partial q_i} = 0
\end{equation}

Insert the friction.
\begin{equation}
\frac{\partial }{\partial t} \frac{\partial L}{\partial \dot{q}_i}
- \frac{\partial L}{\partial q_i}
+ \frac{\partial C}{\partial \dot{q}_i}
= 0
\end{equation}

Solve it for $\ddot{\theta}$ and $\ddot{\phi}$.
\begin{equation}
\begin{split}
0& = \frac{\partial }{\partial t} \frac{\partial L}{\partial \dot{\theta}} - \frac{\partial L}{\partial \theta} + \frac{\partial C}{\partial \dot{q}_i}	\\
& = \frac{\partial }{\partial t} ( \frac{1}{2} m  l^2 2 \dot{\theta}) 
 - [
      \frac{1}{2} m  l^2 2 \sin\theta \cos\theta \dot{\phi}^2 
      - ml ( -f_z \sin\theta + (f_x\cos\phi-f_y\sin(-\phi))(-\cos(\theta))  )
    ]
    + \frac{\partial C}{\partial \dot{q}_i}					\\
%
& =  m  l^2 \ddot{\theta} 
 - m  l^2 \sin\theta \cos\theta \dot{\phi}^2 - m l f_z \sin\theta - ml(f_x\cos\phi - f_y\sin(-\phi))\cos\theta
 + \frac{\partial C}{\partial \dot{q}_i}											\\
%
& =  \ddot{\theta} 
- \sin\theta \cos\theta \dot{\phi}^2 - \frac{\sin\theta}{l} f_z - \frac{\cos\theta}{l}(f_x\cos\phi - f_y\sin(-\phi))
+ \frac{1}{ml^2}\frac{\partial C}{\partial \dot{q}_i}        \\
%
\ddot{\theta}& = \frac{\sin(2\theta)}{2} \dot{\phi}^2 
+ \frac{\sin\theta}{l} f_z + \frac{\cos\theta}{l}(f_x\cos\phi - f_y\sin(-\phi))
- \frac{1}{ml^2}\frac{\partial C}{\partial \dot{q}_i}
\end{split}
\end{equation}

\begin{equation}
\begin{split}
0& = \frac{\partial }{\partial t} \frac{\partial L}{\partial \dot{\phi}} - \frac{\partial L}{\partial \phi} + \frac{\partial C}{\partial \dot{\phi}}	\\
& = \frac{\partial }{\partial t} ( m  l^2 \sin^2\theta \dot{\phi}) 
- (-1) m l (-f_x \sin\phi + f_y \cos(\phi)) \sin(-\theta)
+ \frac{\partial C}{\partial \dot{\phi}}								\\
%
& = m  l^2(\sin^2\theta \ddot{\phi} + 2 \sin\theta \cos\theta \dot{\theta} \dot{\phi})
+ml (-f_x\sin\phi + f_y \cos(\phi)) \sin(-\theta)
+ \frac{\partial C}{\partial \dot{\phi}}	\\
%
& = \ddot{\phi} + 2\frac{\cos\theta}{\sin\theta} \dot{\theta} \dot{\phi}
+ \frac{\sin(-\theta)}{l}(-f_x\sin\phi + f_y \cos\phi)
+ \frac{1}{ml^2\sin^2\theta}\frac{\partial C}{\partial \dot{\phi}}		\\
\ddot{\phi}& = - 2\cot{\theta} \dot{\theta} \dot{\phi} 
+ \frac{\sin(\theta)}{l}(-f_x\sin\phi + f_y \cos\phi)
- \frac{1}{ml^2\sin^2\theta}\frac{\partial C}{\partial \dot{\phi}}
\end{split}
\end{equation}




%% MAIN EQUATIONS DONE

\section{Friction and Drag}

The non generalized \emph{angular} friction power $C_A$ and \emph{drag} friction power $C_D$ are as follows.
For the drag, I will pretend the the surface $A$ and the density $p$ (through volume) is proportional to $l$. The extra components are included in the drag coefficient $c_d$. While the velocity for $C_A$ can be described only through $\theta$ and $\phi$, $C_D$ depends on the velocity of the system and has to include the wind speed.

\begin{align}
  \begin{split}
  C_A &= \frac{1}{2} c_a v^2							                              \\
  C_A &= \frac{1}{2} c_a l^2 (\sin^2\theta \dot{\phi}^2 + \dot{\theta}^2)
  \end{split}
  \\
  \begin{split}
  C_D &= \frac{1}{2} c_{d0} p A (v - w)^3    							            \\
  C_D &= \frac{1}{2} c_{d} \frac{m}{l} l (|v - w|)^3                               \\
  C_D &= \frac{1}{2} c_{d} m l^3 (\frac{{v_w}^2}{l^2})^{\frac{3}{2}}                              \\
  \end{split}
\end{align}


\subsection{Derivates of the frictions for the main equation}
\subsubsection{Angular Friction}
\begin{align}
  \frac{\partial C_A}{\partial \dot{\theta}} &= c_a l^2 \dot{\theta} 	\\
  \frac{\partial C_A}{\partial \dot{\phi}}   &= c_a l^2 \sin^2\theta\dot{\phi} \\
\end{align}

\subsubsection{Drag}

${v_w}^2$ has to be inserted, see \hyperref[ssec:rvelocity]{this previous section (click)}.
\begin{align}
  \begin{split}
  \frac{\partial C_D}{\partial \dot{\theta}}   &=\frac{\partial }{\partial \dot{\theta}} \frac{1}{2} c_{d} m ({v_w}^2)^{\frac{3}{2}}      \\
  & = \frac{3}{4} c_{d} m
  (\frac{{v_w}^2}{l^2})^\frac{1}{2}
  (2 l \cos\theta \cos\phi (l \dot{\theta} \cos\theta \cos\phi - l \dot{\phi} \sin\theta \sin\phi - x_w)
  \\&\phantom{{}={}}+ 2 l \cos\theta  \sin\phi (l \dot{\theta} \cos\theta \sin\phi +l \dot{\phi} \sin\theta \cos\phi - y_w)
  + 2 l \sin\theta (l \dot{\theta} \sin\theta - z_w))
  \end{split}
  \\
  \begin{split}
  \frac{\partial C_D}{\partial \dot{\phi}}   
  & = \frac{1}{2} c_{d} m \frac{\partial (\frac{{v_w}^2}{l^2})^{\frac{3}{2}} }{\partial \dot{\phi}} \\
  & = \frac{3}{4} c_{d} m
  (\frac{{v_w}^2}{l^2})^\frac{1}{2}                    
  \\&\phantom{{}={}}(2 l \sin\theta (l \dot{\phi} \sin\theta (\sin^2\phi + \cos^2\phi + x_w \sin\phi - y_w \cos\phi)))
\end{split}
\end{align}








\section{Multiple Segments}

Here is an attempt to generalize the equation to compute the motion of multiple pendulums concatenated.

\begin{align}
  \begin{split}
    x& = l \sin{\theta} \cos{\phi}\\
    y& = l \sin{\theta} \sin{\phi}\\
    z& = -l \cos{\theta}\\
  \end{split}
  \begin{split}
    \dot{x}& = -l\sin{\theta} \sin{\phi} \dot{\phi} + l\cos{\theta}\cos{\phi} \dot{\theta}\\
    \dot{y}& = l\sin{\theta} \cos\phi \dot{\phi} + l\cos\theta\sin{\phi} \dot{\theta}\\
    \dot{z}& = l\sin\theta \dot{\theta}
  \end{split}
\end{align}

Some partial derivates.

\begin{align}
  \begin{split}
    \frac{\partial x}{\partial \theta}& = x_{n\theta} = l \cos{\theta} \cos{\phi} \\
    \frac{\partial y}{\partial \theta}& = y_{n\theta} = l \cos{\theta} \sin{\phi} \\
    \frac{\partial z}{\partial \theta}& = z_{n\theta} = l \sin{\theta}
  \end{split}
  \\
  \begin{split}
    \frac{\partial \dot{x}}{\partial t} & = \dot{x}_{t}
      = - l\cos{\theta} \sin{\phi} \dot{\theta} \dot{\phi}
      - l\sin{\theta} \cos{\phi} \dot{\phi}^2
      - l\sin{\theta} \sin{\phi} \ddot{\phi}
      \\&\phantom{{}=xn={}}- l\cos{\theta}\sin{\phi} \dot{\theta} \dot{\phi}
      - l\sin{\theta}\cos{\phi} \dot{\theta}^2
      + l\cos{\theta}\cos{\phi} \ddot{\theta}
    \\
    \frac{\partial \dot{y}}{\partial t} & = \dot{y}_{t}
      = l\cos{\theta} \cos\phi \dot{\theta} \dot{\phi}
      - l\sin{\theta} \sin\phi \dot{\phi}^2
      + l\sin{\theta} \cos\phi \ddot{\phi}
      \\&\phantom{{}=xn={}}
      + l\cos\theta\cos{\phi} \dot{\theta} \dot{\phi}
      - l\sin\theta\sin{\phi} \dot{\theta}^2
      - l\cos\theta\sin{\phi} \ddot{\theta}
    \\
    \frac{\partial \dot{z}}{\partial t} & = \dot{z}_{t}
      = l\cos\theta \dot{\theta}^2 + l\sin\theta \ddot{\theta}
  \end{split}
  \\
  \begin{split}
    \frac{\partial \dot{x}}{\partial \dot{\theta}}& = \dot{x}_{\dot{\theta}} 
      = l\cos{\theta}\cos{\phi} \\
    \frac{\partial \dot{y}}{\partial \dot{\theta}}& = \dot{y}_{\dot{\theta}} 
      = l\cos\theta\sin{\phi}\\
    \frac{\partial \dot{z}}{\partial \dot{\theta}}& = \dot{z}_{\dot{\theta}} 
      = l\sin\theta
  \end{split}
  \\
  \begin{split}
    \frac{\partial}{\partial t}\frac{\partial \dot{x}}{\partial \dot{\theta}}& = \dot{x}_{\dot{\theta}t} 
      = -l\sin{\theta}\cos{\phi} \dot{\theta} - l\cos{\theta}\sin{\phi}  \dot{\phi}\\
      \frac{\partial}{\partial t}\frac{\partial \dot{y}}{\partial \dot{\theta}}& = \dot{y}_{\dot{\theta}t} 
      = -l\sin{\theta}\sin{\phi} \dot{\theta} + l\cos{\theta}\cos{\phi}  \dot{\phi}\\
      \frac{\partial}{\partial t}\frac{\partial \dot{z}}{\partial \dot{\theta}}& = \dot{z}_{\dot{\theta}t} 
      = l\cos\theta \dot{\theta}
  \end{split}
\end{align}













\begin{comment}
  
\begin{split}
  \frac{\partial x_n}{\partial \phi_n}& = x_{n\phi}  = l_n \sin{\theta_n} \sin{\phi_n} \phi_n\\
  \frac{\partial y_n}{\partial \phi_n}& = y_{n\phi}  = l_n \sin{\theta_n} \cos{\phi_n} \phi_n\\
  \frac{\partial z_n}{\partial \phi_n}& = z_{n\phi}  = 0
\end{split}
\begin{split}
  \frac{\partial \dot{x}_n}{\partial \phi_n}& = \dot{x}_{n\phi} 
    = -l_n\sin{\theta_n} \cos{\phi_n} \dot{\phi}_n - l_n\cos{\theta_n}\sin{\phi_n} \dot{\theta}_n\\
  \frac{\partial \dot{y}_n}{\partial \phi_n}& = \dot{y}_{n\phi} 
    = -l_n\sin{\theta_n} \sin\phi_n \dot{\phi}_n + l_n\cos\theta_n\cos{\phi_n} \dot{\theta}_n\\
  \frac{\partial \dot{z}_n}{\partial \phi_n}& = \dot{z}_{n\phi} 
    = 0
\end{split}
\begin{split}
  \frac{\partial \dot{x}_n}{\partial \dot{\phi}}& = \dot{x}_{n\dot{\phi_n}} 
    = -l_n\sin{\theta_n} \sin{\phi_n} \\
  \frac{\partial \dot{y}_n}{\partial \dot{\phi}}& = \dot{y}_{n\dot{\phi_n}} 
    = l_n\sin{\theta_n} \cos\phi_n\\
  \frac{\partial \dot{z}_n}{\partial \dot{\phi}}& = \dot{z}_{n\dot{\phi_n}} 
    = 0
\end{split}
\end{comment}




The updated lagrangian.

\begin{align}
  \begin{split}
    T &= \frac{1}{2} m v^2                 \\
    &= \frac{1}{2} ((\sqrt{m}\dot{x})^2 + (\sqrt{m}\dot{y})^2 + (\sqrt{m}\dot{z})^2)                 \\
    &= \frac{1}{2} ((\sqrt{m_a}\dot{x}_a + \sqrt{m_b}\dot{x}_b \dots)^2 + (\sqrt{m_a}\dot{y}_a + \sqrt{m_b}\dot{y}_b \dots)^2 + (\sqrt{m_a}\dot{z}_a + \sqrt{m_b}\dot{z}_b \dots)^2)       
  \end{split} \\
  \begin{split}
    V &= -(((m_a + m_b) x_a + m_b x_b) x_f) + ((m_a + m_b) y_a + m_b y_b) y_f) + ((m_a + m_b) z_a + m_b x_b) z_f))
  \end{split}
\end{align}

\begin{align}
  \begin{split}
    L &= \frac{1}{2} (m_a\dot{x}_a^2 + 2 \sqrt{m_a + m_b} \dot{x}_a \dot{x}_b + m_b\dot{x}_b^2 
    +m_a\dot{y}_a^2 + 2 \sqrt{m_a + m_b} \dot{y}_a \dot{y}_b + m_b\dot{y}_b^2
    \\&\phantom{{}=\frac{1}{2} (}+m_a\dot{z}_a^2 + 2 \sqrt{m_a + m_b} \dot{z}_a \dot{z}_b + m_b\dot{z}_b^2)
    \\&\phantom{{}={}}+ ((m_a + m_b) x_a + m_b x_b) x_f + ((m_a + m_b) y_a + m_b y_b) y_f + ((m_a + m_b) z_a + m_b x_b) z_f
  \end{split}
\end{align}

For the topmost $\theta$ and $\phi$, this will result in.

\begin{align}
\frac{\partial }{\partial t} \frac{\partial L}{\partial \dot{q}_i} - \frac{\partial L}{\partial q_i} = 0
\end{align}












\section {Numpy ODE}

By entering $\theta$, $\dot{\theta}$, $\phi$ and $\dot{\phi}$ into an ODE, they motion can be solved numerically.

\begin{lstlisting}[language=Python]

def pend(y, t, mass, gravity, L, c):
    theta, d_theta, phi, d_phi = y
    dydt = [d_theta,
            np.square(d_phi) * np.sin(theta) * np.cos(theta) - gravity/L * np.sin(theta) - c/mass * d_phi,
            d_phi,
            (-2) * d_theta * d_phi / np.tan(theta) - c/mass * d_phi
            ]
    return dydt

mass = 10
gravity = 10
L = 0.2
friction = 8

# intial values
y0 = [np.pi/2.5, 0.0, 0.0, 4]

# time values
t = np.linspace(0, 20, 400)

# create the solver
sol = scipy.integrate.odeint(pend, y0, t, args=(mass, gravity, L, friction))
\end{lstlisting}

\newpage
\tableofcontents
\end{document}


